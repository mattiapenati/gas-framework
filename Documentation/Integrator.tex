\documentclass[10pt]{amsart}
\usepackage[italian]{babel}
%\usepackage[T1]{fontenc}
\usepackage[latin1]{inputenc}
%\usepackage{geometry}
%\geometry{a4paper}
\usepackage[parfill]{parskip}
\usepackage{graphicx}
\usepackage{color}
\usepackage{graphics}
\usepackage{amsmath,amssymb,amsthm}
\usepackage{booktabs}
\usepackage[font=small,format=hang,labelfont={sf,bf}]{caption}

%\usepackage[dvips]{epsfig}
%\DeclareGraphicsExtensions{.ps,.eps}
%\usepackage{fancyhdr}

\linespread{1.1}

\title[Formule di Quadratura]{Integrazione Numerica}
\author[A. Fasc\`i, D. Ferrarese, M. Penati]{Alfonso Fasc\`i, Davide Ferrarese, Mattia Penati}
\dedicatory{Relazione per il corso di Programmazione Avanzata\\[1mm]
			\emph{Ingegneria Matematica - Laurea Specialistica}}

\begin{document}
\maketitle

\section{Introduzione}
Sia $f$ una funzione a valori reali integrabile sull'intervallo $[a,b]$. Calcolare esplicitamente l'integrale $I(f)= \int_a^b f(x) dx$ potrebbe essere molto difficile o addirittura impossibile. Una qualsiasi formula esplicita adatta a fornire un'approssimazione di $I(f)$ \`e detta {\it formula di quadratura} o {formula di integrazione numerica}. Un esempio pu\`o essere ottenuto sostituendo $f$ con una sua approssimante $f_n$, dipendente dall'intero $n \ge 0$ e calcolando $I(f_n)$ anzich\`e $I(f)$. Ponendo $I_n(f) = I(f_n)$, si ottiene:
\begin{eqnarray}
\label{appr_num}
I_n(f)=\int_a^b f_n(x) dx & , & n \ge 0
\end{eqnarray}
L'approssimante $f_n$ deve essere facilmente integrabile. Per tale motivo, un naturale approccio consiste nell'utilizzare l'interpolante polinomiale di Lagrange $f_n=\Pi_n f$ su un'insieme di $n+1$ nodi distinti ${x_i}$, con $i=0,\dots, n$. Adottando questa strategia, dalla (\ref{appr_num}) segue che:
\begin{equation}
\label{appr_lag_num}
I_n(f)=\sum_{i=0}^n f(x_i)\int_a^b l_i(x)dx
\end{equation}
dove $l_i$ \`e il polinimio caratteristico di Lagrange, con $n$ gradi di libert\`a, associato ai nodi $x_i$. Si pu\`o notare che la (\ref{appr_lag_num}) \`e una particolare istanza della seguente formula di quadratura:
\begin{equation}
\label{form_quad}
I_n(f)=\sum_{i=0}^n \alpha_i f(x_i)
\end{equation}
dove i coefficienti $\alpha_i$ della combinazione lineare sono dati da $\int_a^b l_i(x) dx$. La formula (\ref{form_quad}) \`e una somma pesata dei valori di $f$ nei punti $x_i$, con $i=0, \dots, n$. Questi punti sono detti {\it nodi} della formula di quadratura, mentre i numeri $\alpha_i \in \mathbb{R}$ sono detti {\it coefficienti} o {\it pesi}. Poich\`e $f$ \`e stata sostituita con la sua interpolante polinomiale, la (\ref{form_quad}) \`e chiamata {\it formula di quadratura interpolatoria}. Il {\it grado di esattezza} di una formula di quadratura \`e definito come il massimo intero $r \ge 0$ per il quale:
\begin{eqnarray}
I_n(f)=I(f), & & \forall f \in \mathbb{P}_r
\end{eqnarray} 
Una qualsiasi formula di quadratura che utilizza $n+1$ nodi distinti ha grado di esattezza maggiore o uguale a $n$. Infatti, se $f \in \mathbb{P}_n$, allora $\Pi_n f = f$ e dunque $I_n(\Pi_n f)=I(\Pi_n f)$.
\section{Formule di Newton-Cotes}
Tali formule si basano sull'interpolazione di Lagrange con nodi equispaziati nell'intervallo di integrazione $[a,b]$. Fissato $n \ge 0$, denotiamo i nodi di quadratura con $x_k = x_0 + kh$, $k = 0, \dots, n$. Le formule del punto medio, dei trapezi e di Cavalieri-Simpson sono speciali istanze delle formule di Newton-Cotes, avendo considerato rispettivamente $n = 0$, $n = 1$, $n = 2$. In generale, definiamo:\\
$\bullet$ {\it formule chiuse}, quelle per cui $x_0 = a$, $x_n = b$ e $h = \dfrac{b-a}{n}\,(n \ge 1)$;\\
$\bullet$ {\it formule aperte}, quelle per cui $x_0 = a+h$, $x_n = b-h$ e $h = \dfrac{b-a}{n+2}\,(n \ge 0)$.\\
Un'importante propriet\`a delle formule di Newton-Cotes \`e il fatto che i pesi di quadratura $\alpha_i$ dipendono in modo esplicito solamente da $n$ e $h$, non dall'intervallo di integrazione $[a,b]$. Per verificare questa propriet\`a nel caso delle formule chiuse, introduciamo il cambiamento di variabili $x = \Psi (t) = x_0 + th$. Osservando che $\Psi(0) = a$, $\Psi(n) = b$ e $x_k = a + kh$, otteniamo:
\begin{equation}
\dfrac{x-x_k}{x_i-x_k} = \dfrac{a+th-(a+kh)}{a+ih-(a+kh)} = \dfrac{t-k}{i-k}.
\end{equation}
Dunque, se $n \ge 1$:
\begin{eqnarray}
l_i(x) = \prod_{k = 0,k \ne i}^n \dfrac{t-k}{i-k} = \varphi_i (t) & & 0 \le i \le n.
\end{eqnarray}
Si ottiene la seguente espressione per i pesi di quadratura:
\begin{equation}
\alpha_i = \int_a^b l_i(x)dx = \int_0^n \varphi_i (t) hdt = h\int_0^n \phi_i (t) dt,
\end{equation}
dalla quale si ricava:
\begin{eqnarray}
I_n(f) = h \sum_{i=0}^n \omega_i f(x_i), & & \omega_i = \int_0^n \varphi_i (t) dt. 
\end{eqnarray}
I coefficienti $\omega_i$ non dipendono da $a,b,h$ e $f$, ma solamente da $n$, dunque possono essere tabulati {\it a priori}. Nel caso particolare delle formule chiuse, i polinomi $\varphi_i$ e $\varphi_{n-i}$, con $i = 0, \dots, n$, per la propriet\`a di simmetria hanno il medesimo integrale e dunque anche i corrispondenti pesi $\omega_i$ e $\omega_{n-i}$ si equivalgono. Per tale motivo, mostriamo in Tabella \ref{pesi} solo la prima met\`a dei pesi.
\begin{table}[hlr]
\begin{tabular}{|ccccccc|}
\hline
n & 1 & 2 & 3 & 4 & 5 & 6  \\
\hline
\vspace{0.5pt}$\omega_0$ & $\frac{1}{2}$ & $\frac{1}{3}$ & $\frac{3}{8}$ & $\frac{14}{45}$ & $\frac{95}{288}$ & $\frac{41}{140}$ \vspace{0.5pt}\\
\vspace{0.5pt}$\omega_1$ & 0 & $\frac{4}{3}$ & $\frac{9}{8}$ & $\frac{64}{45}$ & $\frac{375}{288}$ & $\frac{216}{140}$ \vspace{0.5pt}\\
\vspace{0.5pt}$\omega_2$ & 0 & 0 & 0 & $\frac{24}{45}$ & $\frac{250}{288}$ & $\frac{27}{140}$ \vspace{0.5pt}\\
\vspace{0.5pt}$\omega_3$ & 0 & 0 & 0 & 0 & 0 & $\frac{272}{140}$ \vspace{0.5pt}\\
\hline
\end{tabular}
\begin{tabular}{|ccccccc|}\hline
n & 0 & 1 & 2 & 3 & 4 & 5 \\
\hline
\vspace{0.5pt} $\omega_0$ & 2 & $\frac{3}{2}$ & $\frac{8}{3}$ & $\frac{55}{24}$ & $\frac{66}{20}$ & $\frac{4277}{1440}$\vspace{0.5pt}\\
\vspace{0.5pt} $\omega_1$ & 0 & 0 & $-\frac{4}{3}$ & $\frac{5}{24}$ & $-\frac{84}{20}$ & $-\frac{3171}{1440}$\vspace{0.5pt}\\
\vspace{0.5pt} $\omega_2$ & 0 & 0 & 0 & 0 & $\frac{156}{20}$ & $\frac{3934}{1440}$\vspace{0.5pt}\\
\hline
\end{tabular}
\caption{Pesi delle formule chiuse (sinistra) e aperte (destra) di Newton-Cotes\label{pesi}}
\end{table}
Per aumentare l'accuratezza di una formula di quadratura interpolatoria, potrebbe non essere conveniente incrementare il valore di $n$. In tal modo, si possono presentare gli stessi inconvenienti dell'interpolazione di Lagrange su nodi equidistanti. Per esempio, si osserva che i pesi delle formule di Newton-Cotes chiuse con $n=8$ hanno segno opposto, comportando instabilit\`a numeriche dovute a errori di arrotondamento e rendendo cos\`i tale formula poco utile nella pratica. Ci\`o accade per tutte le formule di Newton-Cotes con un numero di nodi maggiore di 8.

\section{Integrazione Numerica Bidimensionale}
Sia $\Omega$ un dominio limitato in $\mathbb{R}^2$ con bordo sufficientemente regolare. Consideriamo il problema di approssimare l'integrale $I(f) = \int_{\Omega} f(x,y) dxdy$, dove $f$ \'e una funzione continua su $\bar{\Omega}$. In particolare, sia $\Omega$ un poligono convesso sul quale si assegna una triangolazione $\mathcal{T}_h$, tale che $\bar{\Omega} = \bigcup_{T \in \mathcal{T}_h} T$, dove il parametro $h > 0$ \`e il massimo tra le lunghezze dei lati in $\mathcal{T}_h$. Esattamente come nel caso monodimensionale, le regole di quadratura interpolatoria sono del tipo:
\begin{equation}
\int_{\Omega} f(\mathbf{z}_j) = \sum_{j=1}^N \omega_j f(\mathbf{z}_j),
\end{equation}
dove $\mathbf{z}_j=(x_j,y_j)$ e $\omega_j$ sono rispettivamente i punti e i pesi della formula di integrazione numerica. Sia ora $\mathbb{P}_d$ lo spazio vettoriale finito dimensionale di grado massimo $d$, cio\`e $\mathbb{P}_d =$ span\{$x^n y^m, m+n \le d$\}, e sia $N = \rm{dim}\,\mathbb{P}_d = \frac{1}{2} (d+1)(d+2)$ la dimensione di tale spazio. Consideriamo inoltre il triangolo per cui $x\ge -1$, $y\ge -1$ e $x+y \le 0$. I polinomi $x^n y^m$ sono notoriamente mal coondizionati, perci\`o sar\`a necessario descrivere $\mathbb{P}_d$ con una base pi\`u ragionevole. Per tale motivo utilizzeremo i polinomi ortogonali di Kornwinder-Dubiner ${g_{m,n}}$ (\cite{}):
\begin{equation}
g_{m,n} (\mathbf{z}) = P_m^{0,0} \left(\frac{x}{1-y}\right)(1-y)^m \, P_n^(2m+1,0) (y),
\end{equation}
dove $P_n^{\alpha,\beta}$ sono i polinomi di Jacobi con pesi $(\alpha,\beta)$ e grado $n$. Si pu\`o notare che dato un insieme $N$ di punti ${\mathbf{z}_j}$ nel triangolo, si ottengono i pesi di Newton-Cotes risolvendo il sistema $N \times N$:
\begin{eqnarray}
\label{korndub}
\sum_{j=1}^N \omega_j g_{m,n}(\mathbf{z}_j) = \int g_{m,n}(\mathbf{z}) d\mathbf{z}, & & \forall g_{m,n} \in \mathbb{P}_d. 
\end{eqnarray}
Per costruzione, i pesi di Newton-Cotes e i punti ${\mathbf{z}_j}$ forniscono una formula di quadratura che integra le $N$ funzioni di base, e perci\`o:
\begin{eqnarray}
\sum_{j=1}^N \omega_j g(\mathbf{z}_j) = \int g (\mathbf{z}) d\mathbf{z}, & & \forall g \in \mathbb{P}_d.
\end{eqnarray}
Poich\`e qualsiasi insieme di $N$ punti di quadratura deve soddisfare l'equazione (\ref{korndub}), i pesi per qualsiasi formula di quadratura di questo tipo devono essere pesi di Newton-Cotes. 
\end{document}
